%%%%%%%%%%%%%%%%%%%%%%%%%%%%%%%%%%%%%%%%%
% Twenty Seconds Resume/CV
% LaTeX Template
% Version 1.0 (14/7/16)
%
% This template has been downloaded from:
% http://www.LaTeXTemplates.com
%
% Original author:
% Carmine Spagnuolo (cspagnuolo@unisa.it) with major modifications by 
% Vel (vel@LaTeXTemplates.com)
%
% License:
% The MIT License (see included LICENSE file)
%
%%%%%%%%%%%%%%%%%%%%%%%%%%%%%%%%%%%%%%%%%

%----------------------------------------------------------------------------------------
%	PACKAGES AND OTHER DOCUMENT CONFIGURATIONS
%----------------------------------------------------------------------------------------

\documentclass[letterpaper,UTF8]{twentysecondcv} % a4paper for A4


%中文支持
\usepackage{xeCJK}
\usepackage{xltxtra}
\XeTeXlinebreaklocale "zh" 
% \XeTeXlinebreakskip = 0pt plus 1pt 

\setmainfont[Mapping=tex-text]{Times New Roman} % rm
\setsansfont[Mapping=tex-text]{Arial}           % sf
\setmonofont{Courier New}                       % tt



% Command for printing skill progress bars
\newcommand\skills{ 
~
	\smartdiagram[bubble diagram]{
        \textbf{\large{算法分析}}\\\textbf{\large{与设计}},
        \textbf{\large{关系型}}\\\textbf{\large{数据库}},
        \textbf{\large{设计模式}},
        \textbf{~~\large{\LaTeX}~~},
        \textbf{\large{机器学}}\\\textbf{\large{习算法}},
        \textbf{\large{数据}}\\\textbf{\large{挖掘}},
        \textbf{\large{大数据}}\\\textbf{\large{原理与技术}},
        \textbf{\large{数学}}\\\textbf{\large{建模}}
    }
}

\interests{{MarkDown/3.5},{C,C++/4.7},{Latex/3.8},{HTML,CSS,JS/4.2},{Python/4.5},{Java/5.5}}

%----------------------------------------------------------------------------------------
%	 PERSONAL INFORMATION
%----------------------------------------------------------------------------------------

% If you don't need one or more of the below, just remove the content leaving the command, e.g. \cvnumberphone{}

\profilepic{img/avatar.jpg} % Profile picture

\cvname{\qquad 黄君扬} % Your name
\cvjobtitle{\qquad \ 本科三年级\ 在读} % Job title/career
\cvlinkedin{https:github.com/sorahjy}
\cvnumberphone{+86 180-1900-2473} % Phone number
\cvsite{https://blog.sorahjy.com} % Personal website
\cvmail{sorahjy@gmail.com} % Email address

%----------------------------------------------------------------------------------------

\begin{document}
\makeprofile % Print the sidebar

%----------------------------------------------------------------------------------------
%	 EDUCATION
%----------------------------------------------------------------------------------------
\section{Education}

\begin{twenty} % Environment for a list with descriptions
	\twentyitem
    	{预计毕业 \\ 2019.7}
        {计算机科学与技术}
        {}
        {上海理工大学 光电信息与计算机工程学院}
        {GPA: 4.03, 排名:1/107}
	% \twentyitem
 %    	{2009 - 2013}
 %        {BEng., Computer Engineering}
 %        {\href{http://www.unipune.ac.in/}{University of Pune}}
 %        {Pune, Maharashtra, India}
 %        {GPA: 4.0, First Class with Distinction}
	%\twentyitem{<dates>}{<title>}{<organization>}{<location>}{<description>}
\end{twenty}


\section{Awards}
\begin{twenty}
	\twentyitem
    	{2016 - 2018}
        {上海理工大学学习优秀奖学金}
        {\textbf{{\color{violet}{\normalsize{连续四次一等奖}}}}}
        {}
        {}
    \twentyitem
        {2017}
        {上海市奖学金}
        {\textbf{{\color{violet}{\normalsize{市级表彰}}}}}
        {}
        {}
    \twentyitem
        {2017.4}
        {第八届蓝桥杯Java软件开发省赛(上海)}
        {\textbf{{\color{violet}{\normalsize{一等奖}}}}}
        {}
        {}
    \twentyitem
        {2017.5}
        {第八届蓝桥杯Java软件开发全国总决赛}
        {\textbf{{\color{violet}{\normalsize{二等奖}}}}}
        {}
        {}
    \twentyitem
        {2017.9}
        {2017年全国大学生数学建模竞赛上海赛区本科组}
        {\textbf{{\color{violet}{\normalsize{三等奖}}}}}
        {}
        {}
    \twentyitem
        {2017.11}
        {第42届ACM-ICPC亚洲区域赛(青岛)}
        {\textbf{{\color{violet}{\normalsize{银奖}}}}}
        {}
        {}
    \twentyitem
        {2017.12}
        {2017年APMCM亚太地区大学生数学建模竞赛}
        {\textbf{{\color{violet}{\normalsize{二等奖}}}}}
        {}
        {}
    \twentyitem
        {2018.2}
        {2018年美国大学生数学建模竞赛}
        {\textbf{{\color{violet}{\normalsize{二等奖}}}}}
        {}
        {}
    \twentyitem
        {2018.4}
        {第三届中国高校计算机团体程序设计天梯赛}
        {\textbf{{\color{violet}{\normalsize{上海市特等/全国三等奖}}}}}
        {}
        {}
    \twentyitem
        {2018.4}
        {第九届蓝桥杯Java软件开发省赛(上海)}
        {\textbf{{\color{violet}{\normalsize{一等奖}}}}}
        {}
        {}
\end{twenty}

%----------------------------------------------------------------------------------------
%	 EXPERIENCE
%----------------------------------------------------------------------------------------


\section{Experience}

\begin{twenty} % Environment for a list with descriptions
	\twentyitem
    	{2017.10 - \\2018.1}
        {酒店管理系统}
        {}
        {\href{https://github.com/sorahjy/HotelAstolfo}{https://github.com/enihsyou/HotelAstolfo}}
        {和同学共同开发。本人负责数据库设计和DAO设计。
        {%\begin{itemize}
        %\item
        %\end{itemize}}
        }}
    \twentyitem
    	{2017.10 - \\ 至今}
        {融合卷积降噪自动编码器的协同过滤推荐系统}
        {}
        {}
        {导师实验室的课题,本人工作如下:{\begin{itemize}
        \item 参与研究、讨论和设计课题中的一种模型,来解决冷启动问题。
        \item 整理实验数据并校验文章内容。
        \end{itemize}}
        }
        
    \twentyitem
   		{2017.12 - \\ 2018.3}
        {一种便携的集成身份认证方法}
        {}
        {\href{https://github.com/sorahjy/Identity-Authentication-WeAPP}{https://github.com/sorahjy/Identity-Authentication-WeAPP}}
        {个人项目。该项目的理念是以一种平台的方式,将传统的地理定\\
        位,动态密码,人脸识别三项技术相融合,并结合微信用户唯一\\
        的openid,提供多维度、多层次的身份认证服务。
        {\begin{itemize}
        \item 部署方便,成本低廉,无需额外硬件设备。
        \item 认证迅速,容易维护,易推广,易扩展。
        \item 采用MVVM模式,部分页面采用了SPA(单页面应用)技术。
    \end{itemize}}
        }
\end{twenty}

\section{Certificates}
\begin{twenty}
    \twentyitem
        {}
        {英语能力 CET-4 590 | CET-6 534}
        {}
        {}
        {}
    \twentyitem
        {}
        {上海市计算机等级考试二级C语言 优秀(99/100分)}
        {}
        {}
        {}
	%\twentyitem{<dates>}{<title>}{<location>}{<description>}
\end{twenty}
\end{document} 
